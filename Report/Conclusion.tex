\section{Conclusion}
\label{sec:conclusion}

To help predict churn risk of pupils from the subscription to Whizz online tutorial system, we have proposed a mixture based model which identifies behaviour clusters associated with distinguished levels of churn rate. Moreover, we fit this model into a Markov chain setting  to allow us understand temporal customer behavioural path. The sequential modelling processes have been formulated into a scalable pipeline that can be easily reused, updated and extended for many applications.

We describe customer behaviours as a result of the state-cluster-observation generative process. The mixture based model can infer clusters from observed behaviours, and clusters will be grouped to form states by bespoke risk appetite. We have formed 4 states from Whizz's data where the ``riskiest'' state has a churn rate 5.5 time higher then the population average. Besides, state transition analysis implies that the intention of churn either tends to grow gradually over the lifecycle, or is purely external and irrelevant to customer experience at Whizz. The model trained from observed data can then be used to make prediction on the churn risk of active pupils, and also analyse the potential causes of such risk. There appears no major overfitting issue.

I'm very confident about this mixture based behavioural model approach in churn prediction. There's room for future work. First, we have not discussed in details about the feature selection and feature engineering, though the model itself can absorb any number of features. Selecting independent, informative features may greatly improve the clustering outcome and prediction accuracy. There is room to mine useful information from Whizz's pupils activity database. Second, it is possible to extend the analysis about how feature impacts churn risk. One direction is to find a way to compare feature importance, which translates into measuring importance of dimensions in a multivariate setting. This will be very useful to help the downstream CRM team to prioritise retention strategies.  
