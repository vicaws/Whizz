\documentclass[a4paper]{article}
\usepackage[a4paper, total={6in, 9in}]{geometry}
\usepackage{color}
\setlength{\hoffset}{-0.5in}\hoffset-0.5in
\setlength{\textwidth}{17.8cm}
\usepackage{hyperref}
\usepackage{amsmath, amsfonts, amsthm, amssymb}
\usepackage{bm} % access bold symbols in maths mode
\usepackage[super]{nth}
\usepackage{verbatim}
\usepackage{stmaryrd}
\usepackage{fancyhdr}
\usepackage{color}
\usepackage{graphicx}
\usepackage{subfigure}
%\usepackage{subfig}
\usepackage{wrapfig}
\graphicspath{{Figure/}} % Set the default folder of all figures as "Figure"

\usepackage{cleveref}
\linespread{1}
\newfam\msbfam
\def\Bbb#1{\fam\msbfam\relax#1}

% Enable multi-columns
\usepackage{multicol, caption}
\setlength{\columnsep}{0.5cm}

\newenvironment{Figure} % Enable inserting figures under multi-column environment
  {\par\medskip\noindent\minipage{\linewidth}}
  {\endminipage\par\medskip}
\newenvironment{Table} % Enable inserting tables under multi-column environment
   {\par\bigskip\noindent\minipage{\columnwidth}\centering}
   {\endminipage\par\bigskip}

\topmargin = 0pt
\voffset = -20pt
\addtolength{\textheight}{2cm}
\newtheorem{theorem}{Theorem}[section]
\newtheorem{exa}{Example}[section]
\newtheorem{corollary}[theorem]{Corollary}
\newtheorem{lemma}[theorem]{Lemma}
\newtheorem{proposition}[theorem]{Proposition}

\theoremstyle{definition}
\newtheorem{definition}[theorem]{Definition}
\newtheorem{remark}[theorem]{Remark}
\newtheorem{notation}[theorem]{Notation}
\newtheorem{assumption}[theorem]{Assumption}
\newtheorem{conjecture}[theorem]{Conjecture}

\usepackage{mathtools}
\DeclarePairedDelimiter\ceil{\lceil}{\rceil} % Ceil function

\newcommand{\ind}{1\hspace{-2.1mm}{1}} %Indicator Function
\newcommand{\I}{\mathtt{i}}
\newcommand{\D}{\mathrm{d}}
\newcommand{\E}{\mathrm{e}}
\newcommand{\RR}{\mathbb{R}}
\newcommand{\sgn}{\mathrm{sgn}}
\newcommand{\atanh}{\mathrm{arctanh}}
\def\equalDistrib{\,{\buildrel \Delta \over =}\,}
\numberwithin{equation}{section}
\def\blue#1{\textcolor{blue}{#1}}
\def\red#1{\textcolor{red}{#1}}

%%%%%%%%%%%%%%%%%%%%%%%%%%%%%%%%%%%%%%%%%%%%%%%%%%%%
%\text{}\newpage
%%%%%%%%%%%%%%%%%%%%%%%%%%%%%%%%%%%%%%%%%%%%%%%%%%%%
\setcounter{tocdepth}{4}
\newpage
%\newpage
%\include{ThesisNotations}
%%%%%%%%%%%%%%%%%%%%%%%%%%%%%%%%%%%%%%%%%%%%%%%%%%%%
\fancyhead{}
\fancyfoot{}
\pagestyle{fancy} 
%\fancyhead{\sffamily\small \thepage}
%\fancyhead{\sffamily\small \nouppercase{\rightmark}}
\fancyhead[RO,LE]{\sffamily\small \thepage}
\fancyhead[LO,RE]{\sffamily\small \nouppercase{\rightmark}}
\renewcommand{\headrulewidth}{0.4pt}
\renewcommand{\footrulewidth}{0.0pt}

% The matrix with vertical and horizontal labels
% \usepackage{kbordermatrix}
\usepackage{etoolbox}
\let\bbordermatrix\bordermatrix
\patchcmd{\bbordermatrix}{8.75}{4.75}{}{}
\patchcmd{\bbordermatrix}{\left(}{\left[}{}{}
\patchcmd{\bbordermatrix}{\right)}{\right]}{}{}

% Tables
\usepackage{booktabs}