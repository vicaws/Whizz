\section{Clustering Analysis}
\label{sec:clustering}


Imbalanced data - the reason we don't mak them balanced is because we do not want to twist numerical value of churn rate. For example, if we balance the data set perfectly, then the population churn rate will become 50\%. After we do the clustering task, it is not straightforward to transform back the cluster churn rate for the balanced data set to that for the original imbalanced data set.


\subsection{Distributional Modelling of Features}

empirical distributional properties require simply Gaussian mixtures

correlation analysis enables to separate a few features to be fitted independently

\subsection{Fitting Baysian Gaussian Mixture Model}

brief description of EM-algorithm and Baysian inference

\subsection{Assessing Churn Probability}

(start of the supervised learning, because of the usage of churn label)

compute cluster churn rate

interpret churn probability of individual pupil

\subsection{Feature Impact}

\subsection{Markov States Temporal Transition Analysis}

\subsubsection{Defining Markov States}

\subsubsection{Transitional Analysis}

calculate transition probability

visualise transition (matrix + Sankey plot) \cite{hmm2015}